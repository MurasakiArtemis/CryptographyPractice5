\documentclass[titlepage, 12pt]{article}

\usepackage[utf8]{inputenc}			%Caracteres
\usepackage{amssymb, amsmath, amsfonts, mathtools}		%Matemáticas
\usepackage[bookmarks,hidelinks]{hyperref}	%Links en el ToC
\usepackage[usenames,dvipsnames]{color}		%Colores
\usepackage{listings}				%Código
\usepackage{url}				%Links web
\usepackage[hypcap]{caption}	%Imagenes
\usepackage{float}				%Imágenes

\usepackage{pdflscape}

%Ajustar márgenes
\topmargin=-0.45in
\evensidemargin=0in
\oddsidemargin=0in
\textwidth=6.5in
\textheight=9.0in
\headsep=0.25in

\numberwithin{equation}{section}%Número de ecuaciones (#Sección, #Ecuación)
\numberwithin{figure}{section}%Número de imágenes (#Sección, #Imagen)
\numberwithin{table}{section}%Número de imágenes (#Sección, #Tabla)

\lstdefinestyle{customc}{
	language=C,
	frame=single, % Single frame around code
	basicstyle=\small\ttfamily, % Use small true type font
	keywordstyle=[1]\color{Blue}\bf, % C functions bold and blue
	keywordstyle=[2]\color{Purple}, % C function arguments purple
	keywordstyle=[3]\color{Blue}\underbar, % Custom functions underlined and blue
	identifierstyle=, % Nothing special about identifiers                                         
	commentstyle=\usefont{T1}{pcr}{m}{sl}\color[rgb]{0.0,0.4,0.0}\small, % Comments small dark green courier font
	stringstyle=\color{Purple}, % Strings are purple
	showstringspaces=false, % Don't put marks in string spaces
	tabsize=2, % 2 spaces per tab
	% Put standard C functions not included in the default language here
	morekeywords={},
	% Put C function parameters here
	morekeywords=[2]{},
	% Put user defined functions here
	morekeywords=[3]{},
	morecomment=[l][\color{Blue}]{...}, % Line continuation (...) like blue comment
	numbers=left, % Line numbers on left
	firstnumber=1, % Line numbers start with line 1
	numberstyle=\tiny\color{Blue}, % Line numbers are blue and small
	stepnumber=5, % Line numbers go in steps of 5
	extendedchars=true,
	inputencoding=utf8,
	literate={á}{{\'a}}1 {é}{{\'e}}1 {í}{{\'i}}1 {ó}{{\'o}}1 {ú}{{\'u}}1 {ñ}{{\~n}}1,
	} 

\lstdefinestyle{customcpp}{
	language=C++,
	frame=single, % Single frame around code
	basicstyle=\small\ttfamily, % Use small true type font
	keywordstyle=[1]\color{Blue}\bf, % C++ functions bold and blue
	keywordstyle=[2]\color{Purple}, % C++ function arguments purple
	keywordstyle=[3]\color{Blue}\underbar, % Custom functions underlined and blue
	identifierstyle=, % Nothing special about identifiers                                         
	commentstyle=\usefont{T1}{pcr}{m}{sl}\color[rgb]{0.0,0.4,0.0}\small, % Comments small dark green courier font
	stringstyle=\color{Purple}, % Strings are purple
	showstringspaces=false, % Don't put marks in string spaces
	tabsize=4, % 4 spaces per tab
	% Put standard C++ functions not included in the default language here
	morekeywords={std,string,cout,endl,ifstream,ofstream,ios, srand, time, nullptr},
	% Put C++ function parameters here
	morekeywords=[2]{iostream,fstream},
	% Put user defined functions here
	morekeywords=[3]{},
	morecomment=[l][\color{Blue}]{...}, % Line continuation (...) like blue comment
	numbers=left, % Line numbers on left
	firstnumber=1, % Line numbers start with line 1
	numberstyle=\tiny\color{Blue}, % Line numbers are blue and small
	stepnumber=5, % Line numbers go in steps of 5
	extendedchars=true
	inputencoding=utf8,
	literate={á}{{\'a}}1 {é}{{\'e}}1 {í}{{\'i}}1 {ó}{{\'o}}1 {ú}{{\'u}}1 {ñ}{{\~n}}1,
}

\lstdefinestyle{customlisp}{
	language=lisp,
	frame=single, % Single frame around code
	basicstyle=\small\ttfamily, % Use small true type font
	keywordstyle=[1]\color{Blue}\bf, % Lisp functions bold and blue
	keywordstyle=[2]\color{Purple}, % Lisp function arguments purple
	keywordstyle=[3]\color{Blue}\underbar, % Custom functions underlined and blue
	identifierstyle=, % Nothing special about identifiers                                         
	commentstyle=\usefont{T1}{pcr}{m}{sl}\color[rgb]{0.0,0.4,0.0}\small, % Comments small dark green courier font
	stringstyle=\color{Purple}, % Strings are purple
	showstringspaces=false, % Don't put marks in string spaces
	tabsize=4, % 4 spaces per tab
	% Put standard C++ functions not included in the default language here
	morekeywords={},
	% Put C++ function parameters here
	morekeywords=[2]{},
	% Put user defined functions here
	morekeywords=[3]{},
	morecomment=[l][\color{Blue}]{...}, % Line continuation (...) like blue comment
	numbers=left, % Line numbers on left
	firstnumber=1, % Line numbers start with line 1
	numberstyle=\tiny\color{Blue}, % Line numbers are blue and small
	stepnumber=5, % Line numbers go in steps of 5
	extendedchars=true
	inputencoding=utf8,
	literate={á}{{\'a}}1 {é}{{\'e}}1 {í}{{\'i}}1 {ó}{{\'o}}1 {ú}{{\'u}}1 {ñ}{{\~n}}1,
}

\title{Binary Fields, AES and Operation Modes}
\author{Eron Romero Argumedo\\Erwin Hernandez Garcia\\3CV1}
\date{October 17, 2016}

\newcommand{\imagen}[4][]{
	\begin{figure}[H]
		\centering
		\includegraphics[#1]{#2}
		\caption{#3}
		#4
	\end{figure}
}

\newcommand{\codescript}[4][]{
	\begin{itemize}
		\item[]\lstinputlisting[caption=#4,label=#3,style=custom#2, #1]{#3.#2}
	\end{itemize}
}

\newcommand{\delimitCodeScript}[6][]{
	\begin{itemize}
		\item[]\lstinputlisting[lastline=#6,firstline=#5, caption=#4,label=#3#5#6,style=custom#2, #1]{#3.#2}
	\end{itemize}	
}

\begin{document}
	\maketitle
	\pagenumbering{roman}
	\tableofcontents
	\listoffigures
	\newpage
	\pagenumbering{arabic}
	\section{Theory}
	
	\subsection{Evariste Galois}
	
	Evariste Galois was a mathematical genius known nowadays for the development of the group theory. \medskip

	Many of its constructions (today called Galois group, Galois fields and Galois theory) remain fundamental concepts in modern algebra.\cite{galois}
	
	
	\subsection{Rijndael and AES}
	
	AES (Advances Encryption Standard) started as an initiative to replace the old Data Encryption Standard (DES) and triple-DES. \medskip
	
	Rijndael is an algorithm made by Joan Daemen and Vincent Rijmen. It was selected as a candidate for the AES. On 2 October, 2000, NIST (National Institute of Standards and Technology) officially announced that Rijndael, without modifications, would become the AES. Some factors that gave Rijndael a quick acceptance were that it is royalty-free and that it can be implemented easily on a wide range of platforms. \cite{rijndael}
	
	\section{Functions}
	\subsection{Section 1}
		\codescript{lisp}{Section1Code/aes-implementation}{Obtaining the keys for AES}
		The AES implementation used was changed because the C implementation that was found did not work with variable key size.
	
	\subsection{Analysing the algorithm}
		
		\delimitCodeScript{c}{Section2CodeInverse/EuclidesPolinomios}{Inverse}{44}{84}
		The inverse function receives two values, a polynomial $a \in GF(2^{t})$ and an irreducible polynomial $m$ of degree t. The output of this function is $a^{-1}$ in the binary field $GF(2^{t})$.
		

	
	\subsection{Modes of operation}
	\subsubsection{Key generation}
		\delimitCodeScript{cpp}{Section2Code/Functions}{Key generation}{12}{23}
		The key generation creates a 16bytes (128bits) random key, then saves them in a file with the specified name.
	\subsubsection{ECB}
		\delimitCodeScript{cpp}{Section2Code/Functions}{ECB mode}{65}{66}
		\delimitCodeScript{cpp}{Section2Code/Functions}{AES}{127}{134}
		The plaintext is ciphered independently of the other blocks
	\subsubsection{CBC}
		\delimitCodeScript{cpp}{Section2Code/Functions}{CBC mode}{82}{97}
		If the solicited operation is encrypt, then an XOR operation is made with the plaintext and the previous ciphertext, the result is then encrypted.
		On the other hand, if the decryption is solicited, first the ciphertext is decrypted and then the XOR is made with this result and the previous ciphertext
	\subsubsection{CTR}
		\delimitCodeScript{cpp}{Section2Code/Functions}{CTR mode}{99}{106}
		The IV is encrypted, then an XOR is made with this result and the plaintext, finally the IV is increased by one to use it on the following round

	\section{Screenshots}
		\subsection{Inverse}
		\subsubsection{Usage}
		\imagen[width=\linewidth]{Section2CodeInverse/ModoUso.png}{Usage of the program}{\label{ModoUso}}
		\subsubsection{Results}
		\imagen[width=\linewidth]{Section2CodeInverse/Prueba1.png}{Test with a(x)=8 and m(x)=13}{\label{Prueba1}}
		\imagen[width=\linewidth]{Section2CodeInverse/Prueba2.png}{Test with a(x)=2 and m(x)=b}{\label{Prueba2}}
		\imagen[width=\linewidth]{Section2CodeInverse/Prueba3.png}{Test with a(x)=1b and m(x)=25}{\label{Prueba3}}
		\imagen[width=\linewidth]{Section2CodeInverse/Prueba4.png}{Test with a(x)=f0 and m(x)=11b}{\label{Prueba4}}
	
		\subsection{Modes of operation}
		\subsubsection{Usage}
		\imagen[width=\linewidth]{Usage.png}{Usage of the program}{\label{Usage}}
		\subsubsection{Results}
		\imagen[width=\linewidth]{PlainText.png}{Original plaintext}{\label{Plaintext}}
		\imagen[width=\linewidth]{Encrypt.png}{Ciphertext obtained after encrypt with AES}{\label{Ciphertext1}}
		\imagen[width=\linewidth]{Decrypt.png}{Plaintext obtained after decrypt with AES}{\label{Ciphertext2}}			
	\bibliographystyle{IEEEtran}
	\bibliography{Bibliography}
\end{document}